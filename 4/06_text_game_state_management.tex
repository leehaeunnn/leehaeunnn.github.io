\section{부가 프로젝트: 텍스트 게임 상태 관리 시스템}

본 시스템은 복잡한 상태 전이를 FSM으로 관리했습니다.
목표는 상태 전이 오류를 줄이는 것이었습니다.

\subsection{아키텍처}
상태는 이름, 입장, 퇴장, 전이로 구성했습니다.
전이는 조건 함수로 정의했습니다.

\begin{figure}[H]
    \centering
    \includegraphics[width=0.8\textwidth]{images/4_fsm.png}
    \caption{FSM 구성과 전이 다이어그램}
    \label{fig:fsm}
\end{figure}

\begin{lstlisting}[language=Python]
class GameStateMachine:
    def __init__(self):
        self.states, self.current, self.history = {}, None, []
    def add_state(self, name, on_enter=None, on_exit=None):
        self.states[name] = {'on_enter': on_enter, 'on_exit': on_exit, 'trans': {}}
    def add_transition(self, src, dst, cond):
        self.states[src]['trans'][dst] = cond
    def transition_to(self, nxt, ctx):
        if self.current and self.states[self.current]['on_exit']:
            self.states[self.current]['on_exit'](ctx)
        self.history.append(self.current)
        self.current = nxt
        if self.states[nxt]['on_enter']:
            self.states[nxt]['on_enter'](ctx)
\end{lstlisting}

