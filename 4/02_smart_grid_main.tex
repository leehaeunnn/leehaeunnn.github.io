\section{동적 스마트그리드 통합 시뮬레이터 개발}

\subsection{프로젝트 개요}
 제71회 전국과학전람회 산업 및 에너지 부문에 출품된 '정전을 막는 방법?(스마트그리드 AI 시뮬레이터)'입니다. 실제 도시 환경과 유사한 전력망 시뮬레이션을 통해 이중화 시스템과 에너지 백업 시스템의 효과를 검증할 수 있는 실용적 도구 개발을 목표로 했습니다. 

\subsection{배경과 목표}
스마트그리드 분야에서 도시 규모 전력망을 통합적으로 시뮬레이션할 수 있는 실용적 도구는 아직 부족합니다. 기존 연구들이 주로 이론적 수학 모델 개발에 집중되어 있어 실제 환경에서의 적용 가능성을 검증하기 어려운 상황입니다. 2024년 7월 집중호우로 인해 송전선로가 파손되고 변전소가 침수된 사례는 현재 중앙집중형 전력 시스템의 구조적 취약성을 명확히 보여주었으며, 효과적인 이중화 및 백업 시스템 구축의 시급성을 일깨워주었습니다. 
이에 본 프로젝트는 복잡한 알고리즘 개발보다는 실제 환경을 반영한 스마트그리드 시뮬레이터 구현에 집중하여, 제안된 기술과 알고리즘의 실용성 및 효율성을 효과적으로 검증할 수 있는 시스템 개발을 목적으로 합니다. 전력망 장애 대응을 위한 이중화 시스템과 에너지 백업 시스템을 실제와 유사한 환경에서 구현하고 검증할 수 있는 실용적 도구를 개발하여, 이론과 실제 운영 간의 간극을 메우고자 했습니다.

\subsection{시스템 설계와 구현}
도시 전력망의 건물과 송전선을 그래프 형태로 모델링하여, 각 건물의 발전·소비·저장 특성이 시간대나 날씨에 따라 동적으로 변화하도록 설계했습니다. 시뮬레이터는 다음과 같은 핵심 구성요소를 포함합니다:

\begin{itemize}[itemsep=0pt]
    \item 건물 단위 에너지 모델링 (발전소, 소비자, 프로슈머)
    \item 실시간 날씨 변화 및 돌발 상황 반영
    \item 경제적 요인을 반영한 전력 흐름 분석
    \item Edmonds-Karp 알고리즘 기반 전력 조류 계산
\end{itemize}

\begin{figure}[H]
    \centering
    \includegraphics[width=0.8\textwidth]{images/4_energy_dashboard.png}
    \caption{스마트그리드 통합 시뮬레이터 대시보드}
    \label{fig:energy_dash}
\end{figure}

\subsection{지능형 운영 전략}
전력망 상태를 자동으로 진단하고, 예산 내에서 송전선 보강, 발전소 건설, 에너지 저장 장치 운영 등의 개선 전략을 수행하는 AI 기능을 구현했습니다. 

\paragraph{자동화된 의사결정.} 시스템은 다음과 같은 최적화 전략을 자동으로 수행합니다:
\begin{itemize}[itemsep=0pt]
    \item 과부하 송전선의 용량 증설 우선순위 결정
    \item 정전 취약 지역 식별 및 발전소 최적 입지 선정
    \item 에너지 저장 시스템(ESS) 운영 스케줄링
    \item 실시간 전력 수급 균형 조정
\end{itemize}

\subsection{핵심 구현 기술}
\begin{lstlisting}[language=Python]
class CityGraph:
    def __init__(self):
        self.buildings = []  # Building nodes
        self.power_lines = []  # Power line edges
        
    def apply_edmonds_karp(self):
        """Calculate power flow using max flow algorithm"""
        # Virtual aggregated source and sink
        # Apply network capacity constraints
        # Identify bottlenecks and blackouts
        return flow_result

class PowerSystem:
    def update_battery(self, building):
        """ESS charge/discharge logic"""
        if self.is_peak_time():
            return self.discharge_battery(building)
        elif self.has_surplus_solar():
            return self.charge_battery(building)

    def analyze_grid_status(self):
        """AI-based grid status diagnosis"""
        bottlenecks = self.find_overloaded_lines()
        blackout_areas = self.find_power_shortage()
        return self.suggest_upgrades(bottlenecks, blackout_areas)
\end{lstlisting}

\subsection{프로젝트 성과와 의의}
시뮬레이터는 다양한 상황을 반영하여 전력망의 구조적 문제를 효과적으로 분석할 수 있었습니다. 자동화된 운영 전략은 정전 발생을 줄이고 효율을 높이는 데 도움이 되었으며, 전력망 내 병목 구간이나 취약 지점을 사전에 파악하고 대응 방안을 모색할 수 있었습니다.

실제 도시 전력망 운영과 유사한 환경을 구현하여 전력망 설계, 정책 검토, 교육 등 다양한 분야에서 활용 가능성이 높습니다. 특히 복잡한 변수들이 작동하는 스마트 그리드 환경을 가상으로 실험해볼 수 있다는 점에서 실용적인 가치를 지닙니다.