\section{기술 역량 발전}

\subsection{AI에 대한 인식 변화}

\begin{figure}[h]
    \includegraphics[width=\columnwidth]{images/1_ai_perception_journey.png}
    \caption{AI 활용 단계별 발전}
    \label{fig:ai_perception_journey}
\end{figure}

AI를 단순 도구에서 협업 파트너로 활용하는 방법을 습득했습니다.

\subsubsection{초기: AI는 마법 상자}

처음에는 ‘AI에게 시키면 다 해준다’고 기대했습니다. 그러나 부정확한 결과와 일관성 없는 응답을 겪으며 한계를 확인했습니다. 결국 AI도 적절한 지시가 필요하다는 점을 깨달았습니다.

\subsubsection{중기: AI와의 협업 방법 터득}

\begin{figure}[h]
    \includegraphics[width=\columnwidth]{images/1_ai_collaboration.png}
    \caption{AI 협업 프로세스}
    \label{fig:ai_collaboration}
\end{figure}

프롬프트 엔지니어링의 중요성을 발견했고 반복적인 개선으로 품질을 끌어올렸습니다. AI의 한계를 이해하고 보완하는 방법을 습득했습니다.

\subsubsection{현재: AI를 활용한 문제 해결사}

복잡한 문제를 AI가 처리 가능한 단위로 분해했습니다. 여러 AI 도구를 조합해 솔루션을 설계했고 인간의 창의성과 AI의 처리 능력을 결합했습니다.

\subsection{기술 습득 내용}


\subsubsection{API 통합과 시스템 설계}

처음에는 API가 무엇인지도 몰랐습니다. 문서를 읽는 법부터 시작해 인증과 요청 제한, 에러 처리를 학습했습니다. 여러 API를 조합해 복잡한 시스템을 구축했습니다.

\subsubsection{비동기 프로그래밍과 성능 최적화}

\begin{figure}[H]
    \centering
    \includegraphics[width=0.8\textwidth]{images/1_async_learning.png}
    \caption{비동기 처리 구현 과정}
    \label{fig:async_learning}
\end{figure}

대량 처리를 위해 Promise와 async/await 패턴을 이해했습니다. 병렬 처리와 배치 최적화를 적용했으며 메모리 관리와 성능 프로파일링을 습득했습니다.

\subsection{문제 해결 능력의 진화}

\begin{figure}[H]
    \centering
    \includegraphics[width=0.8\textwidth]{images/1_problem_solving_evolution.png}
    \caption{문제 해결 접근 방식의 변화}
    \label{fig:problem_solving_evolution}
\end{figure}

\subsubsection{체계적인 문제 분석}

이전에는 문제를 만나면 즉시 코딩을 시작했고 난관에 부딪히면 처음부터 다시 작성했습니다. 그 결과 비효율적인 시행착오가 반복됐습니다.

현재는 문제를 작은 단위로 분해하고 각 부분의 해결 방안을 설계합니다. 단계별로 구현하고 테스트한 뒤 전체를 통합해 최적화합니다.

\subsubsection{실패를 통한 학습}

\begin{figure}[H]
    \centering
    \includegraphics[width=0.8\textwidth]{images/1_failure_learning.png}
    \caption{주요 실패와 그로부터 얻은 교훈}
    \label{fig:failure_learning}
\end{figure}

가장 값진 실패는 세 가지였습니다. OCR 접근의 실패로 Vision AI의 가능성을 발견했습니다. 복잡한 UI의 실패는 사용자 중심 디자인의 중요성을 일깨웠습니다. 과도한 기능의 실패는 핵심 가치에 집중하도록 만들었습니다.

\subsection{소통과 협업 능력}

\begin{figure}[H]
    \centering
    \includegraphics[width=0.8\textwidth]{images/1_communication_skills.png}
    \caption{다양한 이해관계자와의 소통}
    \label{fig:communication_skills}
\end{figure}

\subsubsection{비전문가와의 소통}

교사, 학생, 노년층과 소통하며 기술 용어를 일상 언어로 번역하는 법을 익혔습니다. 시각적 자료로 설명하고 사용자 입장에서 생각하는 태도를 유지했습니다.

\subsubsection{피드백 수용과 반영}

\begin{figure}[H]
    \centering
    \includegraphics[width=0.8\textwidth]{images/1_feedback_incorporation.png}
    \caption{사용자 피드백 반영 프로세스}
    \label{fig:feedback_incorporation}
\end{figure}

비판을 개선의 기회로 전환하고 피드백 수집 시스템을 체계화했습니다. 우선순위를 정해 단계적으로 개선했습니다.

\subsection{미래를 향한 비전}

\begin{figure}[H]
    \centering
    \includegraphics[width=0.8\textwidth]{images/1_future_aspirations.png}
    \caption{AI 기술을 통한 미래 목표}
    \label{fig:future_aspirations}
\end{figure}

\subsubsection{단기 목표}

단기 목표는 더 많은 교육 기관에 시스템을 보급하는 것입니다. 다양한 AI 모델을 실험하고 최적화하며 국제 AI 컨퍼런스에도 참가합니다.

\subsubsection{장기 비전}

장기 비전은 AI 교육 스타트업을 창업하는 것입니다. 개발도상국의 교육을 지원하고 AI 윤리와 책임을 연구합니다.

\subsection{프로젝트가 가르쳐준 것}

이 프로젝트를 통해 배운 가장 중요한 교훈은 기술은 목적이 아니라 수단이라는 것입니다. AI가 아무리 발전해도, 그것을 어떻게 활용해 사회 문제를 해결하고 사람들의 삶을 개선할 것인가가 핵심입니다.

또한 혼자서는 할 수 없다는 것도 깨달았습니다. 교사들의 조언, 사용자들의 피드백, 동료들의 도움이 없었다면 이 프로젝트는 불가능했을 것입니다.

앞으로도 AI 기술을 활용해 더 많은 사람들에게 도움을 줄 수 있는 솔루션을 만들어가고 싶습니다. 특히 교육의 기회가 부족한 사람들, 기술의 혜택에서 소외된 사람들을 위한 프로젝트를 계속해나갈 것입니다.