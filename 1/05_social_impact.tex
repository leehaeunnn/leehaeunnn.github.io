\section{사회적 영향과 가치 창출}

\subsection{교육 격차 해소}

\begin{figure}[H]
    \centering
    \includegraphics[width=0.8\textwidth]{images/1_education_gap_reduction.png}
    \caption{지역별 교육 격차 감소 효과}
    \label{fig:education_gap_reduction}
\end{figure}

AI 채점 시스템은 교육의 질적 격차를 줄였습니다. 농촌 지역 학생도 즉각적인 피드백을 받을 수 있었고 과외를 받지 못하는 학생에게 맞춤형 학습을 제공했습니다. 24시간 이용 가능한 학습 도우미로서의 역할도 수행했습니다.

\subsubsection{실제 사례: 충남 시골 학교}

\begin{figure}[H]
    \centering
    \includegraphics[width=0.8\textwidth]{images/1_rural_school_case.png}
    \caption{시골 학교 적용 사례}
    \label{fig:rural_school_case}
\end{figure}

AI 채점 시스템을 통해 교사들이 채점 시간을 절약하여 개별 지도에 더 많은 시간을 할애할 수 있게 되었습니다.

\subsection{디지털 포용성 증진}

\begin{figure}[H]
    \centering
    \includegraphics[width=0.8\textwidth]{images/1_digital_inclusion.png}
    \caption{연령대별 디지털 서비스 접근성 개선}
    \label{fig:digital_inclusion}
\end{figure}

AI 키오스크는 특수교육 대상 학생들의 자립적 주문 능력을 향상시키고 사회적 상호작용에 대한 자신감을 높이는 데 기여할 수 있습니다.

\begin{figure}[H]
    \centering
    \includegraphics[width=0.8\textwidth]{images/1_elderly_independence.png}
    \caption{노년층 사용자의 긍정적 변화}
    \label{fig:elderly_independence}
\end{figure}

\subsubsection{장애인 접근성}

\begin{figure}[H]
    \centering
    \includegraphics[width=0.8\textwidth]{images/1_disability_access.png}
    \caption{장애 유형별 접근성 개선}
    \label{fig:disability_access}
\end{figure}

음성 인터페이스는 접근성을 확장했습니다. 시각 장애인도 독립적으로 주문할 수 있었고 손 사용이 어려운 장애인을 지원했습니다. 인지 장애인을 위해 대화를 단순화해 실수를 줄였습니다.

\subsection{경제적 가치 창출}

\begin{figure}[H]
    \centering
    \includegraphics[width=0.8\textwidth]{images/1_economic_value.png}
    \caption{프로젝트의 경제적 효과}
    \label{fig:economic_value}
\end{figure}

\subsubsection{교육 비용 절감}

AI 시스템 도입으로 초과근무 필요성이 감소하고 학습 지원을 위한 추가 비용이 절감될 수 있습니다.

\subsubsection{생산성 향상}

AI 키오스크를 통해 직원들이 더 많은 학생들을 효율적으로 관리할 수 있고, 주문 처리 속도가 향상되어 대기 시간이 줄어들 것으로 기대됩니다.

\subsection{기술 확산과 오픈소스 기여}

\begin{figure}[H]
    \centering
    \includegraphics[width=0.8\textwidth]{images/1_opensource_contribution.png}
    \caption{오픈소스 프로젝트 기여 현황}
    \label{fig:opensource_contribution}
\end{figure}

\subsubsection{코드 공개와 커뮤니티}

소스코드를 GitHub에 공개하여 다른 개발자들과 공유하고 협업할 수 있는 기회를 만들었습니다.

\subsubsection{교육 자료 제작}

\begin{figure}[H]
    \centering
    \includegraphics[width=0.8\textwidth]{images/1_educational_materials.png}
    \caption{제작한 교육 자료들}
    \label{fig:educational_materials}
\end{figure}

AI 활용 가이드북과 프롬프트 엔지니어링 워크샵 자료를 제작했습니다. 초보자를 위한 튜토리얼 비디오를 만들고 한국어 문서화에도 기여했습니다.

\subsection{미래 비전과 확장 가능성}

\begin{figure}[H]
    \centering
    \includegraphics[width=0.8\textwidth]{images/1_future_vision.png}
    \caption{AI 교육 플랫폼의 미래 로드맵}
    \label{fig:future_vision}
\end{figure}

\subsubsection{다른 과목으로 확장}

영어 작문 첨삭, 과학 실험 보고서 평가, 역사 논술 채점, 프로그래밍 코드 리뷰 등 적용 범위 확대는 [PLAN], 구체 일정과 범위는 [DETAILS REQUIRED]입니다.

\subsubsection{글로벌 확장}

글로벌 확장을 위해 다국어를 지원하고 각국 교육과정에 맞춤화합니다. 국제 교육 기관과 협력하며 UNESCO 교육 프로젝트 참여도 검토합니다.

\subsection{사회적 책임과 윤리}

\begin{figure}[H]
    \centering
    \includegraphics[width=0.8\textwidth]{images/1_ethical_considerations.png}
    \caption{AI 윤리 가이드라인}
    \label{fig:ethical_considerations}
\end{figure}

\subsubsection{개인정보 보호}

학생 데이터는 암호화해 저장하고 최소 정보만 수집했습니다. 데이터 사용의 투명성을 보장하고 보호자 동의 절차를 구현했습니다.

\subsubsection{AI 편향성 방지}

다양한 배경의 테스트 데이터를 사용했습니다. 편향성 검사를 정기적으로 수행했고 인간 교사의 검토 시스템을 운영했습니다. 모델은 지속적으로 개선했습니다.