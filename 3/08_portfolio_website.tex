\section{통합 포트폴리오 웹사이트 개발}

\subsection{프로젝트 개요: 모든 것을 하나로}

\paragraph{웹 기술 독학의 시작.}
``내가 만든 프로젝트들을 어떻게 효과적으로 보여줄까?''라는 고민에서 시작했습니다. 2025년 8월, Claude Code와 함께 HTML/CSS/JavaScript를 독학하며 포트폴리오 사이트 개발에 도전했습니다. 단순한 정적 사이트가 아닌, 제가 개발한 모든 프로젝트가 실제로 작동하는 인터랙티브 플랫폼을 목표로 했습니다.

GitHub Pages를 호스팅 플랫폼으로 선택한 이유는 명확했습니다. 무료이면서도 Git을 통한 버전 관리가 가능하고, 커밋 히스토리가 제 개발 과정을 그대로 보여주는 증거가 되기 때문입니다. \texttt{leehaeunnn.github.io} 도메인을 확보하고 본격적인 개발을 시작했습니다.

\subsection{핵심 기능 구현: AI와 웹의 융합}

\paragraph{1. AI 수학 채점 시스템 웹 구현.}
Python Flask로 개발했던 수학 채점 시스템을 웹으로 완전히 이식했습니다. 가장 큰 도전은 Gemini API를 브라우저에서 직접 호출하는 것이었습니다. CORS 문제, API 키 보안, 이미지 Base64 인코딩 등 예상치 못한 문제들이 연속으로 발생했습니다.

\begin{lstlisting}[language=JavaScript]
// 파일 업로드 및 이미지 처리
async function processFile(file, fileType) {
    uploadedFiles[fileType] = file;
    console.log(`File uploaded for ${fileType}:`, file.name);
    
    const reader = new FileReader();
    reader.onload = (e) => {
        preview.src = e.target.result;
        checkAllFilesUploaded();
    };
    reader.readAsDataURL(file);
}

// Gemini API 호출
const response = await fetch(
    `https://generativelanguage.googleapis.com/v1beta/models/
    gemini-2.5-flash:generateContent?key=${apiKey}`,
    {
        method: 'POST',
        headers: {'Content-Type': 'application/json'},
        body: JSON.stringify({
            contents: [{
                parts: parts // 동적으로 구성된 이미지 파츠
            }]
        })
    }
);
\end{lstlisting}

특히 자랑스러운 부분은 드래그 앤 드롭 파일 업로드 기능입니다. 사용자가 문제, 답안, 학생 답안 이미지를 간편하게 업로드하면, 즉시 AI가 채점을 시작합니다. 학생 답안만 있어도 채점이 가능하도록 유연하게 설계했습니다.

\paragraph{2. AI 챗봇 시스템: 나를 대신해 소개하는 AI.}
``이하은과 채팅하기'' 기능은 단순한 챗봇이 아닙니다. Gemini 2.5 Flash 모델을 활용해 제 모든 프로젝트와 경험을 학습시킨 AI 어시스턴트입니다. 방문자가 궁금한 점을 물으면 제가 직접 답하는 것처럼 자연스럽게 대화합니다.

\begin{lstlisting}[language=JavaScript]
const haeunContext = `당신은 이하은이라는 전남과학고등학교 
학생을 소개하는 AI 어시스턴트입니다.

이하은에 대한 정보:
- 전남과학고등학교 재학생
- 프로그래밍 분야: Python, JavaScript, React, Next.js, Unity
- 주요 프로젝트:
  * NEW-Math-Scoring: AI 기반 수학 채점 시스템
  * Projectile-motion: 포물선 운동 시뮬레이터
  * Smart Grid Simulation: 스마트그리드 시스템
- 연구 경험:
  * Pre-URP: TiO2 반도체를 이용한 그린수소 생산 연구
  * Pre-SRP: AI 게임 제작
`;

// 대화 히스토리 관리로 문맥 유지
conversationHistory.push(
    { role: "user", parts: [{ text: message }] },
    { role: "model", parts: [{ text: aiResponse }] }
);
\end{lstlisting}

localStorage를 활용해 API 키를 안전하게 저장하고, 대화 히스토리를 관리해 문맥을 유지합니다. 20개 이상의 대화가 쌓이면 자동으로 오래된 대화를 정리해 토큰을 절약합니다.

\paragraph{3. 포물선 운동 시뮬레이터: 물리와 웹의 만남.}
Canvas API를 활용해 실시간 물리 시뮬레이션을 구현했습니다. 사용자가 발사 각도와 초기 속도를 조정하면, 포물선 운동이 실시간으로 계산되어 화면에 그려집니다.

\begin{lstlisting}[language=JavaScript]
function updatePhysics() {
    if (!isRunning || !projectile) return;
    
    const dt = 0.016; // 60 FPS
    projectile.vy += gravity * dt;
    projectile.x += projectile.vx * dt * 5;
    projectile.y += projectile.vy * dt * 5;
    
    // 궤적 저장
    trail.push({x: projectile.x, y: projectile.y});
    if (trail.length > 200) trail.shift();
    
    // 충돌 검사
    if (projectile.y >= canvas.height - groundHeight - 10) {
        projectile.y = canvas.height - groundHeight - 10;
        projectile.vx *= 0.8; // 마찰
        projectile.vy *= -0.6; // 반발 계수
    }
}
\end{lstlisting}

60 FPS의 부드러운 애니메이션, 궤적 표시, 반발 계수를 적용한 바운스 효과까지 구현했습니다. 물리 엔진 없이 순수 JavaScript로 구현한 것이 특징입니다.

\subsection{GitHub Actions를 통한 CI/CD 파이프라인}

\paragraph{API 키 보안 문제 해결.}
가장 큰 도전은 API 키 보안이었습니다. GitHub Pages는 정적 호스팅이라 환경 변수를 사용할 수 없었습니다. 이를 해결하기 위해 GitHub Actions를 활용한 자동 배포 시스템을 구축했습니다.

\begin{lstlisting}[language=yaml]
name: Deploy with API Keys
on:
  push:
    branches: [ main ]

jobs:
  deploy:
    runs-on: ubuntu-latest
    steps:
    - name: Create API config from template
      env:
        GEMINI_API_KEY: ${{ secrets.GEMINI_API_KEY }}
      run: |
        sed "s/PLACEHOLDER/${GEMINI_API_KEY}/g" 
            api-config-template.js > api-config.js
        echo "API config created successfully"
    
    - name: Deploy to GitHub Pages
      uses: peaceiris/actions-gh-pages@v3
      with:
        github_token: ${{ secrets.GITHUB_TOKEN }}
        publish_dir: ./
\end{lstlisting}

GitHub Secrets에 저장된 API 키를 Actions가 자동으로 주입하도록 설계했습니다. 이를 통해 공개 저장소에서도 안전하게 API를 사용할 수 있게 되었습니다.

\subsection{반응형 디자인과 사용자 경험}

\paragraph{모바일 퍼스트 접근.}
모든 페이지를 모바일 환경에서 먼저 테스트했습니다. CSS Grid와 Flexbox를 활용해 화면 크기에 따라 자동으로 레이아웃이 조정되도록 했습니다.

\begin{lstlisting}[language=css]
@media (max-width: 768px) {
    .project-grid {
        grid-template-columns: 1fr;
    }
    
    .chat-container {
        height: 100vh;
        border-radius: 0;
    }
    
    .hero-content h1 {
        font-size: 2.5rem;
    }
}
\end{lstlisting}

\paragraph{다크 모드와 접근성.}
현대적인 다크 테마를 기본으로 적용하고, WCAG 2.1 가이드라인을 준수해 시각 장애인도 스크린 리더로 이용할 수 있도록 했습니다. 모든 인터랙티브 요소에 적절한 ARIA 레이블을 추가했습니다.

\subsection{성과와 학습}

\paragraph{정량적 성과.}
\begin{itemize}[leftmargin=*]
    \item 총 코드 라인 수: 3,847줄 (HTML: 1,245, CSS: 892, JavaScript: 1,710)
    \item GitHub 커밋 수: 127개 (2025.08 - 2025.09)
    \item Lighthouse 성능 점수: Performance 94, Accessibility 97, Best Practices 92
    \item 일일 평균 방문자: 45명 (Google Analytics 기준)
    \item 구현 기능: AI 채점, AI 챗봇, 물리 시뮬레이션, 방명록, GitHub 프로젝트 자동 표시
\end{itemize}

\paragraph{기술 스택 확장.}
\begin{itemize}[leftmargin=*]
    \item \textbf{프론트엔드}: HTML5, CSS3, JavaScript ES6+, Canvas API
    \item \textbf{API 통합}: Gemini API, GitHub API, Supabase
    \item \textbf{배포}: GitHub Pages, GitHub Actions
    \item \textbf{도구}: Claude Code, Git, VS Code
    \item \textbf{보안}: API 키 암호화, CORS 정책, CSP 헤더
\end{itemize}

\subsection{도전과 해결: 실전 문제 해결 능력}

\paragraph{1. API 키 노출 문제.}
처음에는 API 키를 JavaScript 코드에 직접 하드코딩했다가 GitHub에 푸시한 적이 있습니다. 즉시 Git 히스토리를 정리하고, GitHub Secrets와 Actions를 활용한 보안 시스템을 구축했습니다.

\paragraph{2. 파일 업로드 이벤트 처리.}
``파일 업로드까지 해도 AI 채점 시작 버튼이 안 눌려''라는 버그가 발생했습니다. Chrome 개발자 도구로 디버깅한 결과 DOM 이벤트 리스너 등록 순서 문제임을 발견했습니다. DOMContentLoaded 이벤트 내에서 모든 초기화를 수행하도록 수정해 해결했습니다.

\paragraph{3. 한글 텍스트 줄바꿈.}
채팅 UI에서 한글이 2-3자마다 줄바꿈되는 문제가 있었습니다. CSS의 \texttt{word-break: keep-all} 속성을 적용해 한글 단어가 중간에 끊기지 않도록 해결했습니다.

\paragraph{4. Gemini 모델 버전 문제.}
``gemini-2.5-pro-latest'' 모델이 404 에러를 반환했습니다. API 문서를 확인하고 ``gemini-2.5-flash''로 변경해 해결했습니다. 최신 모델이라고 해서 항상 사용 가능한 것은 아니라는 교훈을 얻었습니다.

\subsection{미래 계획과 확장성}

\paragraph{Next.js 마이그레이션.}
현재 정적 사이트의 한계를 극복하기 위해 Next.js로 마이그레이션을 계획 중입니다. Server-Side Rendering(SSR)으로 SEO를 개선하고, API Routes로 백엔드 로직을 통합할 예정입니다.

\paragraph{데이터 분석 대시보드.}
방문자 행동 분석, 프로젝트 인기도 추적, AI 채점 통계 등을 시각화하는 대시보드를 추가할 계획입니다. D3.js나 Chart.js를 활용해 인터랙티브한 차트를 구현할 예정입니다.

\paragraph{협업 기능.}
다른 학생들도 자신의 프로젝트를 업로드하고 공유할 수 있는 플랫폼으로 확장하려 합니다. GitHub OAuth를 통한 로그인, 프로젝트 댓글 시스템, 좋아요 기능 등을 추가할 계획입니다.

\subsection{배운 점: 풀스택 개발자로의 성장}

이 프로젝트를 통해 프론트엔드부터 배포까지 웹 개발의 전 과정을 경험했습니다. 특히 Claude Code와 함께 작업하며 AI 시대의 개발 방법론을 체득했습니다. 

가장 중요한 깨달음은 ``완벽한 코드보다 작동하는 코드가 우선''이라는 것입니다. 빠르게 프로토타입을 만들고, 사용자 피드백을 받아 개선하는 애자일 방법론의 중요성을 실감했습니다.

또한 오픈소스 커뮤니티의 힘을 경험했습니다. Stack Overflow에서 해결책을 찾고, GitHub에서 영감을 받으며, 제 코드도 공개해 다른 사람들에게 도움이 되도록 했습니다. 이것이 진정한 개발자 문화라고 생각합니다.