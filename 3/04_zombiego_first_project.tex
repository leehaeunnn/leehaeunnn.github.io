\section{첫 프로젝트: ZombieGo - KAIST SRP 프로그램}

\subsection{ZombieGo 게임 개발}
2024년 KAIST SRP 프로그램에서 메인 프로그래머로 저비 서바이벌 게임을 개발했습니다.

초기 단순 슈팅 게임에서 AI 경로탐색, 시야 제한 시스템 등을 추가하여 총 3000줄 규모의 코드로 확장했습니다.

\subsection{AI 경로탐색 구현}

\paragraph{A* 알고리즘과의 첫 만남.} 
처음에는 좀비가 단순히 플레이어를 향해 직진하도록 만들었습니다. 하지만 테스트 결과 좀비들이 벽에 막혀 비효율적인 움직임을 보였습니다. 장애물을 우회하여 추적하는 기능이 필요했습니다. 

이 문제를 해결하기 위해 처음으로 A* 경로탐색 알고리즘을 공부했습니다. heapq를 사용해 최적 경로를 계산하고, 15픽셀 단위의 그리드로 맵을 분할했습니다. 수십 번의 디버깅 끝에 좀비가 벽을 우회하여 플레이어를 추적하는 기능을 성공적으로 구현했습니다.

\paragraph{좀비에게 '지능'을 부여하다.}
단순한 추적만으로는 부족했습니다. 실제 공포 게임처럼 좀비가 플레이어를 '탐색'하는 행동을 구현하고 싶었습니다:

\begin{itemize}[leftmargin=*]
    \item \textbf{시야 시스템}: 1200픽셀 범위 내 Line of Sight로 시야 확인. 좀비가 플레이어를 발견했음을 확인할 수 있었습니다.
    \item \textbf{마지막 목격 위치 추적}: last\_known\_pos를 저장해 플레이어가 사라져도 마지막 위치 주변을 수색하도록 구현
    \item \textbf{Unstuck 메커니즘}: 좀비가 12프레임 이상 같은 위치에 있으면 우회 경로를 생성. 이 기능 하나 때문에 며칠을 고민했지만, 이를 해결하며 문제 해결 능력이 크게 향상되었습니다.
\end{itemize}

\subsection{Fog of War 시스템 개발}

\paragraph{안개 속의 긴장감 - Fog of War 시스템.}
좀비 게임의 핵심은 '공포'입니다. 하지만 맵 전체가 다 보이면 긴장감이 떨어졌습니다. 그래서 RTS 게임의 '전장의 안개' 개념을 도입하기로 했습니다.

처음에는 단순히 원형 마스크를 씌우려 했지만, 벽 너머까지 보이는 문제가 있었습니다. 이를 해결하기 위해 BFS(너비 우선 탐색) 알고리즘을 적용했습니다. 300픽셀 시야 반경 내에서 벽에 막히지 않은 영역만 밝히는 시스템을 구현하는 데 성공했습니다. 

특히 자랑스러운 부분은 create\_circular\_gradient() 함수로 구현한 부드러운 시야 경계입니다. 갑자기 어두워지는 것이 아니라 점진적으로 어두워지도록 만들어, 실제 랜턴을 든 것 같은 효과를 냈습니다.

\paragraph{캐릭터에 생명을 불어넣다.}
정적인 스프라이트만으로는 생동감이 부족했습니다. 플레이어가 공격할 때 14프레임의 애니메이션이 재생되도록 만들었고, 좀비도 idle과 attack 상태에 따라 다른 애니메이션을 보여주도록 했습니다. 

상자 시스템도 추가했습니다. E키를 누르면 상자를 열 수 있고, JSON 파일에서 아이템 정보를 읽어와 보상을 주도록 했습니다. 빈 상자 발견 시 표시되는 메시지도 게임 디자인의 일부로 구현했습니다.

\begin{figure}[h]
    \includegraphics[width=\columnwidth]{images/3_zombiego.png}
    \caption{ZombieGo 게임 플레이 화면 - 안개 속에서 좀비와 대치하는 긴장감 넘치는 순간}
    \label{fig:zombiego}
\end{figure}

\subsection{성능 최적화}

\paragraph{60 FPS를 지켜라!}
게임이 복잡해질수록 프레임이 떨어지기 시작했습니다. 2400x1800 크기의 거대한 맵에서 좀비 10마리가 동시에 경로를 계산하니 버벅거렸습니다. 

첫 번째 해결책은 이미지 캐싱이었습니다. \_image\_cache 딕셔너리를 만들어 한 번 로드한 이미지는 재사용하도록 했습니다. 두 번째로 fog surface를 50\% 크기로 축소하여 렌더링했습니다. 눈에 띄는 품질 저하 없이 성능을 2배 향상시킬 수 있었습니다.

가장 효과적이었던 것은 좀비의 경로 계산 주기를 조절한 것입니다. 거리에 따라 15-45프레임마다 한 번씩만 재계산하도록 했더니, 플레이어는 차이를 느끼지 못하면서도 성능은 크게 향상되었습니다.

\paragraph{스토리를 담다 - 대화 시스템.}
단순한 액션 게임이 아닌, 스토리가 있는 게임을 만들고 싶었습니다. DialogueManager 클래스를 만들어 Excel 파일(act0.xlsx, act1-1.xlsx 등)에서 대화 스크립트를 읽어오도록 했습니다. 맵이 바뀔 때마다 자동으로 해당 챕터의 대화가 시작되도록 구현했습니다.

대화 진행 상태를 시스템이 정확히 추적하고 있음을 확인했습니다.

\subsection{프로젝트 성과}

A* 알고리즘, BFS, 상태 머신, 이벤트 기반 프로그래밍 등을 실제 게임 개발에 적용했습니다. KAIST SRP 프로그램을 통해 대규모 프로젝트 개발 능력을 향상시켰습니다.

